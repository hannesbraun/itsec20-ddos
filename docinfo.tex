% -------------------------------------------------------
% Daten für die Arbeit
% Wenn hier alles korrekt eingetragen wurde, wird das Titelblatt
% automatisch generiert. D.h. die Datei titelblatt.tex muss nicht mehr
% angepasst werden.

% Titel der Arbeit auf Deutsch
\newcommand{\hsmatitelde}{Distributed Denial of Service}

% Weitere Informationen zur Arbeit
\newcommand{\hsmaort}{Offenburg}    % Ort
\newcommand{\hsmaautorvname}{Jonas} % Vorname(n)
\newcommand{\hsmaautornname}{Kienzle} % Nachname(n)
\newcommand{\hsmadatum}{07.08.2020} % Datum der Abgabe
\newcommand{\hsmajahr}{2020} % Jahr der Abgabe
\newcommand{\hsmafirma}{XYZ} % Firma bei der die Arbeit durchgeführt wurde
\newcommand{\hsmabetreuer}{Prof. Dr. rer. nat. Stephan Trahasch, Hochschule Offenburg} % Betreuer an der Hochschule
\newcommand{\hsmafakultaet}{EMI} % Fakultät
\newcommand{\hsmastudiengang}{AI} % Studiengangsabkürzung. 

% -------------------------------------------------------
% Abstract

% Kurze (maximal halbseitige) Beschreibung, worum es in der Arbeit geht auf Deutsch
\newcommand{\hsmaabstractde}{DDoS-Angriffe werden in der heutigen Zeit zu einem immer größeren Problem. Mit der immer weiterschreitenden Digitalisierung entstehen täglich neue potenzielle Geräte, welche zum Angriff verwendet werden können. Gleichermaßen wächst auf der Seite der Software auch immer die Angriffsfläche und somit die Zahl der möglichen Einfallstore für diese Art von Angriffen. Im Folgenden wird ein Überblick über DoS- und DDoS-Attacken im Allgemeinen gegeben. Ebenfalls werden mögliche Angriffsmethoden wie zum Beispiel HTTP Slow Post oder SYN-Flooding vorgestellt. Schließlich werden noch ein paar der möglichen Schutzmaßnahmen aufgezeigt.}
