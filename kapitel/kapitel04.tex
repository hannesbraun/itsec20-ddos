\chapter{Angriffsziele}
\label{chap:kapitel4}

Führt ein Angreifer eine DDoS Attacke durch kann dies mehr, als nur ein Ziel haben. Im Folgenden sollen einige Ziele erklärt werden.

\section{Degradation of Service}

Bei  dem Ziel Degradation of Service geht nicht darum ein gesamtes System oder einen Server komplett lahm zu legen, sondern nur darum, die Qualität des angegriffenen Servers zu reduzieren. Damit werden beim Angriff also nicht Ressourcen des Servers restlos aufgebraucht, aber ein bedeutenden Teil davon. Versucht nun ein Nutzer reguläre Anfragen an den Server zu schicken muss mit erhöhten Wartezeiten und eventuell mit fehlgeschlagenen Verbindungen gerechnet werden. Eine Degradation of Service ist vor allem auch dann ein vielversprechendes Ziel, wenn der Betreiber des angegriffenen Servers seine Ressourcen bei einem Anbieter nach Auslastung bezahlt. Durch erhöhte Auslastung beim Server während des Angriffs entstehen dann beträchtliche Mehrkosten beim Anbieter. Weiterhin gibt es den Vorteil, dass es schwer sein kann den Angriff von normalem erhöhtem Traffic zu unterscheiden.

\section{Denial of Service}

Hierbei geht es wirklich darum einen Server komplett außer Funktion zu bringen. Ein gutes Beispiel hierfür ist der Ping of Death. Mit diesem Angriff konnte bis vor einigen Jahren ein modifizierter Ping an Unix Systeme geschickt werden, der einen Systemabsturz zur Folge hatte. Damit war der Server von außen nicht mehr erreichbar und musste neu gestartet werden. Gerade für viel besuchte Websites, wie YouTube oder Netflix könnte so etwas sehr teuer werden, wenn der Fehler erst spät gefunden wird.

\section{Ablenkung}

Es kann auch möglich sein, dass DDoS Angriffe allein zur Ablenkung durchgeführt werden. Es ist gut möglich, dass ein Angreifer ein Ziel mit einer anderen Angriffsvariante verfolgt und gleichzeig einen DDoS Angriff auf einen anderen Server des gleichen Unternehmens durchführt. Dadurch, dass der DDoS Angriff Aufmerksamkeit und Personal an sich bindet hofft der Angreifer darauf, dass der eigentliche Angriff gar nicht erst entdeckt wird oder  weniger Personal dafür verfügbar ist etwas gegen eigentliche Bedrohung zu tun.

\section{Erpressung}

Ein weiteres Ziel kann eine Erpressung des Betreibers des Servers sein. Hierbei wird meist ein kleiner gestartet. Zeitgleich wird dem Betreiber des Servers eine Nachricht geschickt die mit einem wesentlich stärkeren Angriff droht und eine Geldsumme für das Abbrechen des Angriffs fordert. Diese Forderungen beziehen sich normalerweise auf Kryptowährungen wie Bitcoin, damit der Angreifer auch mit dem erhalten des Geldes anonym bleiben kann. Experten raten allerdings immer davon ab solchen Forderungen nachzugeben.