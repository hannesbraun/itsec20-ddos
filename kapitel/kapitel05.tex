\chapter{Distributed DoS Angriffe (DDoS)}
\label{chap:kapitel5}

Distributed DoS Angriffe sind DoS Angriffe welche von mehreren Systemen aus gleichzeitig erfolgen. Das bringt den Vorteil einer höheren Bandbreite und mehr Rechenleistung bei einem Angriff. Dadurch kann bedeutend mehr Traffic an ein Angriffsziel geschickt werden, als bei einfachen DoS Angriffen. Allerdings ergibt sich für einen Angreifer, der eine DDoS Attacke durchführen will das Problem, dass man erstmal viele Systeme braucht, welche einen Angriff durchführen können. Für einen Angreifer wird das Kaufen der Systeme keine Option sein, da er wahrscheinlich über keine großen finanziellen Ressourcen verfügt. Im Folgenden werden Möglichkeiten vorgestellt, welche ein Angreifer nutzen könnte, um mehr Systeme für einen DDoS Angriff zu erhalten.

\section{Botnetz}

Ein Botnetz ist ein Netzwerk aus mehreren Computern. Die Computer in so einem Netzwerk werden Zombies genannt und führen eine Software aus, welche ohne das Wissen des Besitzers auf Kommandos von außen reagiert. Mit den richtigen Kommandos kann dann ein DDoS Angriff vorbereitet und durchgeführt werden.

\section{DDoS Malware}

Eine weitere Möglichkeit, um einen DDoS Angriff vorzubereiten ist das Schreiben einer DDoS Malware. Eine solche Malware ist darauf ausgelegt sich erst als Wurm zu Verbreiten und später zu einem vorher bestimmten Zeitpunkt den DDoS Angriff zu starten. Ein Beispiel für so einen Angriff ist Mydoom. Dieser Wurm hatte die Aufgabe auf infizierten Systemen eine Backdoor zu hinterlassen und einen DDoD Angriff gegen die SCO Group am 1.2.2004 zu starten. Zu diesem Zeitpunkt waren geschätzte 1.000.000 Computer mit dem Virus infiziert. Mydoom ist bis heute einer der massivsten DDoS Angriffe und der sich am schnellsten verbreitende E-Mail Wurm.