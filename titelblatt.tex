% -------------------------------------------------------
% In dieser Datei sollten eigentlich keine Veränderungen mehr
% notwendig sein.
% -------------------------------------------------------

\thispagestyle{empty}

% Fakultät
% -------------------------------------------------------
\newcommand{\hsmafakultaetlangde}{Fakultät Elektrotechnik, Medizintechnik und Informatik}

\newcommand{\hsmastudienganglangde}{Angewandte Informatik}
\newcommand{\hsmatypde}{HAUSARBEIT}

\newcommand{\hsmakoerperschaftde}{Hochschule für Technik, Wirtschaft und Medien Offenburg}

\newcommand{\hsmaautorbib}{Jonas Kienzle und Hannes Braun} % Autor Nachname, Vorname
\newcommand{\hsmaautor}{Jonas Kienzle und Hannes Braun} % Autor Vorname Nachname


\newcommand{\hsmatyp}{\hsmatypde}%
\newcommand{\hsmathesistype}{Praktikum IT-Security}%
\newcommand{\hsmakoerperschaft}{\hsmakoerperschaftde}%
\newcommand{\hsmastudiengangname}{Studiengang \hsmastudienganglangde}%
\newcommand{\hsmastudienganglang}{\hsmastudienganglangde}%
\newcommand{\hsmatitel}{\hsmatitelde}%
\newcommand{\hsmatutor}{Dozent}%
\newcommand{\hsmafakultaetlang}{\hsmafakultaetlangde}%
\newcommand{\hsmalistoftables}{Tabellenverzeichnis}%
\newcommand{\hsmalistoffigures}{Abbildungsverzeichnis}%
\newcommand{\hsmalistings}{Quellcodeverzeichnis}%
\newcommand{\hsmaindex}{Index}%
\newcommand{\hsmaabbreviations}{Abkürzungsverzeichnis}%   
\selectlanguage{ngerman}


% Daten in die Standard-Felder von KOMA-Script eintragen
\titlehead{\hsmatyp\ in\  \hsmastudienganglang}
\subject{}
\title{\hsmatitel}
\author{\hsmaauthor}
\date{\small{\hsmadatum}}

% Daten für das fertige PDF-Dokument
\hypersetup{
  pdftitle={\hsmatitel},  % Titel des Dokuments
  pdfauthor={\hsmaautor},              % Autor
  pdfsubject={\hsmatyp\ in\ \hsmastudienganglang},                % Thema
  pdfkeywords={\hsmatitel}         % Schlüsselworte
}

\newlength{\bindekorrektur}
\newlength{\seitenanfang}
\newlength{\seitenbreite}
  
\setlength{\bindekorrektur}{-46mm}   % Korrektur der horizontalen Position
\setlength{\seitenanfang}{0mm}       % Korrektur der vertikalen Position
\setlength{\seitenbreite}{297mm}

%\noindent \includegraphics[width=7cm, left]{hso.png}\hfill \includegraphics[width=2cm, right]{edeka.png} \\
\captionsetup[figure]{labelformat=empty}
\noindent 
\begin{figure}
  \includegraphics[width=10cm,center]{hso.jpg}
  \caption[]{}
\end{figure}
\captionsetup[figure]{labelformat=simple}
% Titel der Arbeit
\begin{textblock*}{128mm}(41mm,\seitenanfang + 62mm) % 4,5cm vom linken Rand und 6,0cm vom oberen Rand
  \centering\Large\sffamily
  \vspace{12mm} % Kleiner zusätzlicher Abstand oben für bessere Optik
  \textbf{\hsmatitel}
\end{textblock*}%

% Name
\begin{textblock*}{\seitenbreite}(\bindekorrektur,\seitenanfang + 108mm)
  \centering\large\sffamily
  \hsmaautor
\end{textblock*}

% Thesis
\begin{textblock*}{\seitenbreite}(\bindekorrektur,\seitenanfang + 130mm)
  \centering\large\sffamily
  \textbf{\hsmatyp}\\
  \begin{small}\hsmathesistype \end{small}\\
  \vspace{6mm}
  \hsmastudiengangname
\end{textblock*}

% Fakultät
\begin{textblock*}{\seitenbreite}(\bindekorrektur,\seitenanfang + 165mm)
  \centering\large\sffamily
  \hsmafakultaetlang\\
  \vspace{2mm}
  \hsmakoerperschaft
\end{textblock*}

% Datum
\begin{textblock*}{\seitenbreite}(\bindekorrektur,\seitenanfang + 190mm)
  \centering\large 
  \textsf{\hsmadatum}
\end{textblock*}

% Betreuer
\begin{textblock*}{\seitenbreite}(\bindekorrektur,\seitenanfang + 240mm)
  \centering\large\sffamily
  \hsmatutor \\
  \vspace{2mm}
  \hsmabetreuer\\
\end{textblock*}

\cleardoublepage

% Abstract
\thispagestyle{empty}
\textsf{\large\textbf{Zusammenfassung}}
\subsubsection*{\hsmatitelde}\hsmaabstractde

